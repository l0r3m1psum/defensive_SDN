% echo report.tex | entr -s 'latex report && dot -Teps topo.dot >topo.eps && gnuplot report.gp && bibtex report && latex -interaction=batchmode report && dvips -q report && gs -q -dNOPAUSE -dBATCH -sDEVICE=pdfwrite -sOutputFile=report.pdf report.ps'
\documentclass[twocolumn]{article}

\usepackage{graphics}

\author{Diego Bellani}
\date{2022}
\title{Report of Project 5\protect\\ for\protect\\ Computer Network Performance}

\pagenumbering{gobble}

\begin{filecontents}[nosearch,overwrite,noheader]{topo.dot}
graph {
	rankdir = BT;
	{
		node [shape = plaintext];
		h1 [label=<h<SUB>1</SUB>>];
		h2 [label=<h<SUB>2</SUB>>];
		h3 [label=<h<SUB>3</SUB>>];
		hm [label=<h<SUB>m</SUB>>]
	}
	A -- B -- C -- A;
	// B -- D;
	h1 -- A;
	h2 -- B;
	// D -- h3;
	h3 -- C;
	hm -- A;
	h3 -- D -- B [style=dashed];
	{rank = same; A; B; hm}
	{rank = same; h1; h2}
	{rank = same; C; D}
	{rank = max; h3}
}
\end{filecontents}

\begin{filecontents}[nosearch,overwrite,noheader]{report.gp}
################################################################################
set terminal eps
set output 'h1.eps'
set yrange[0:0.08]
set ylabel 'end-to-end delay (ms)'
set xlabel 'Time (s)'
set key left

stats 'h1.txt' using ($2 > 2 ? 0 : $2/2) nooutput
f(x) = m*x + b
fit f(x) 'h1.txt' using 0:($2 > 2 ? STATS_mean : $2/2) via m,b

plot 'h1.txt' using ($2/2) with linespoint linewidth 3 pointtype 7 title 'F1', \
	f(x) title 'Best fit' linewidth 3
################################################################################
reset
set terminal eps
set output 'h2.eps'
set yrange[0:0.12]
set ylabel 'end-to-end delay (ms)'
set xlabel 'Time (s)'
set key left

stats 'h2.txt' using ($2 > 2 ? 0 : $2/2) nooutput
f(x) = m*x + b
fit f(x) 'h2.txt' using 0:($2 > 2 ? STATS_mean : $2/2) via m,b

plot 'h2.txt' using ($2/2) with linespoint linewidth 3 pointtype 7 title 'F2', \
	f(x) title 'Best fit' linewidth 3
################################################################################
reset
set terminal eps
set output 'controller.eps'
set yrange[0:0.0012]
set ylabel 'Utilization'
set xlabel 'Time (s)'
set key left

stats 'controller.txt' using 4 nooutput

array names[STATS_blocks]
names[1] = "s1 h1"
names[2] = "s1 h4"
names[3] = "s1 s2"
names[4] = "s1 s3"
names[5] = "s2 h2"
names[6] = "s2 s1"
names[7] = "s2 s3"
names[8] = "s2 s4"
names[9] = "s3 h3"
names[10] = "s3 s1"
names[11] = "s3 s2"
names[12] = "s4 s2"
# FIXME: Why this gives me a warning?
plot for [k=1:STATS_blocks] 'controller.txt' index k using 4 title names[k] \
	with linespoint pointsize 0.7
\end{filecontents}

\begin{filecontents}[nosearch,overwrite,noheader]{report.bib}
@inproceedings {
	mininet,
	author = {Lantz, Bob and Heller, Brandon and McKeown, Nick},
	title = {A Network in a Laptop: Rapid Prototyping for Software-Defined Networks},
	year = {2010},
	isbn = {9781450304092},
	publisher = {Association for Computing Machinery},
	address = {New York, NY, USA},
	url = {https://doi.org/10.1145/1868447.1868466},
	doi = {10.1145/1868447.1868466},
	booktitle = {Proceedings of the 9th ACM SIGCOMM Workshop on Hot Topics in Networks},
	articleno = {19},
	numpages = {6},
	keywords = {emulation, open-flow, rapid prototyping, virtualization, software defined networking},
	location = {Monterey, California},
	series = {Hotnets-IX}
}
@article {
	openflow,
	author = {McKeown, Nick and Anderson, Tom and Balakrishnan, Hari and Parulkar, Guru and Peterson, Larry and Rexford, Jennifer and Shenker, Scott and Turner, Jonathan},
	title = {OpenFlow: Enabling Innovation in Campus Networks},
	year = {2008},
	issue_date = {April 2008},
	publisher = {Association for Computing Machinery},
	address = {New York, NY, USA},
	volume = {38},
	number = {2},
	issn = {0146-4833},
	url = {https://doi.org/10.1145/1355734.1355746},
	doi = {10.1145/1355734.1355746},
	journal = {SIGCOMM Comput. Commun. Rev.},
	month = {mar},
	pages = {69–74},
	numpages = {6},
	keywords = {virtualization, ethernet switch, flow-based}
}
@book {
	ryu,
	author = {RYU project team},
	title = {RYU SDN Framework},
	publisher = {Web},
	year = {2014},
	pages = {286},
	month = {feb}
}
@inproceedings{
	openvswitch,
	title={Extending networking into the virtualization layer.},
	author={Pfaff, Ben and Pettit, Justin and Amidon, Keith and Casado, Martin and Koponen, Teemu and Shenker, Scott},
	booktitle={Hotnets},
	year={2009}
}
@inproceedings {
	sdn-bandwidth,
	author = {Megyesi, P\'{e}ter and Botta, Alessio and Aceto, Giuseppe and Pescap\`{e}, Antonio and Moln\'{a}r, S\'{a}ndor},
	title = {Available Bandwidth Measurement in Software Defined Networks},
	year = {2016},
	isbn = {9781450337397},
	publisher = {Association for Computing Machinery},
	address = {New York, NY, USA},
	url = {https://doi.org/10.1145/2851613.2851727},
	doi = {10.1145/2851613.2851727},
	booktitle = {Proceedings of the 31st Annual ACM Symposium on Applied Computing},
	pages = {651–657},
	numpages = {7},
	keywords = {floodlight, software defined networks, mininet, available bandwidth, OpenFlow, network operating system}, location = {Pisa, Italy},
	series = {SAC '16}
}
\end{filecontents}

\begin{document}

\maketitle

\begin{abstract}
This project consisted in simulating an attack on a particular network topology
and reacting to it (via SDN). We were also required to take measurements of the
utilization of all links and the end-to-end delay of all legitimate flows for
the entire duration of the experiment.
\end{abstract}

\section{Project}
We were given the network topology shown in figure~\ref{fig:topo}. The nodes
$h_1$, $h_2$, $h_3$ and $h_m$ are the hosts, the nodes $A$, $B$, $C$ and $D$ are
the switches, the solid lines represent the links and the dashed lines the links
which can be dynamically provisioned.

The only requirement for this network was that the bandwidth of the link between
$A$ and $C$ had to be higher than one of the link between $B$ and $C$.

The way in which the attack was performed and the way in which we had to react
to it was as follows.
%
\begin{description}
\item{Time $t_0$} $h_1$ and $h_2$ send two traffic flows, $F1$ and $F2$
respectively, to $h_3$. Due to the bandwidth limitation stated above the
controller chooses path $AC$ for $F1$ and $BAC$ for $F2$.
\item{Time $t_1$} $h_m$\footnote{$h_m$ is always considered as a malicious host,
all of its flows are not legitimate.} sends a traffic flow $F3$ to $h_1$ to
pollute the flow table of A.
\item{Time $t_2$} as a consequence the controller redirects $F2$ from $BAC$ to
$BC$.
\item{Time $t_3$} $h_m$ sends traffic flow $F4$ to $h_3$ to pollute the flow
table of C (this is in addition to $F3$).
\item{Time $t_4$} as a consequence the controller asks for a new path to connect
$h_2$ and $h_3$ i.e. $BD$.
\end{description}

In addition to simulating this scenario we also have to measure the utilization
of all links and the end-to-end of all legitimate flows delay for the entire
duration of the experiment.

\begin{figure}
\centering
\resizebox{.5\columnwidth}{!}{\includegraphics{topo}}
\caption{Network topology used for the project.}
\label{fig:topo}
\end{figure}

\section{Implementation}
To create the network topology we had to use Mininet~\cite{mininet}. Open
vSwitch~\cite{openvswitch} was used as the implementation of the virtual switch
in the network. OpenFlow\footnote{In particular we used version 1.0 of this
protocol due to its simplicity with respect to the newer
versions.}~\cite{openflow} was the protocol used for SDN and finally
Ryu~\cite{ryu} was the framework used to implement the controller.

The bandwidth limits on the various links were set through Mininet using Linux's
traffic control (\texttt{tc(8)}) capabilities.

To generate the traffic flows of $h_1$ and $h_2$ \texttt{ping(1)} was
used which conveniently gave us the roundtrip delay, which was used to derive
the end-to-end delay.

\subsection{Measuring Utilization}
Measuring the utilization of all the link proved to be more challenging than we
were expecting but with a bit of ingenuity we managed to do it.

OpenFlow doesn't provide any way to get the the bandwidth of a port. The closest
thing that it provides is a way to ask for the features of a port, among which
there it its supported rate (e.g. 10 Megabit/s full-duplex). Saddly because we
were using a virtual switch this advertised rates don't make sense in this
context. In fact measuring the bandwidth whith \texttt{iperf(1)} gave us a
bandwith of 45 Gigabit/s which is much higher than what the switch said to
support: 10 Gigabit/s.

Using \texttt{iperf(1)} during the experiment was not possible because it
interfered to much with the legitimate traffic in the network altering the
measurements.

To be able to measure the utilization of a link in a given time interval we need
two numbers: the total bandwidth of that link and how many bytes were
transmitted in that time interval. We can easly get the second one thanks to
OpenFlow's \texttt{OFPST\_PORT} request, which gives to the controller some
statistics about the ports of a switch. For the second one we decided to assume,
as in another paper~\cite{sdn-bandwidth}, that the total bandwidth of all links
was known to the controller at all times.

\subsection{Simulating the Attack}
To simulate the attack, and respect all the routing requirements at the same
time, we decided to create a routing algorithm that used the current utilization
of a link a the cost of that edge for the shortest-path algorithm. To simulate
the attack we simply lowered the available bandwidth of the links in such a way
that the controller was forced to use the path described above for $F1$ and
$F2$.

The bandwidth for the links between the switches is reported in
table~\ref{tab:band}. The bandwidth for the links between hosts and switches is
always 1 Gigabit/s since it doesn't matter for the correct routing and the
attack purposes.

\begin{table}
\centering
\begin{tabular}{c|cccc}
      & $AB$ & $BC$ & $CA$ & $BD$\\
\hline
$t_0$ & 10   & 4    & 10   & -\\
$t_1$ & 1    & 4    & 10   & -\\
$t_2$ & 1    & 4    & 10   & -\\
$t_3$ & 1    & 1    & 10   & -\\
$t_4$ & 1    & 1    & 10   & 10
\end{tabular}
\caption{Table reporting the bandwidth of the links between the switches at all
times of the experiment reported in Megabit/s.}
\label{tab:band}
\end{table}

\section{Results}
The graph for end-to-end delay of flow $F1$ and $F2$ can be seen in
figures~\ref{fig:f1} and~\ref{fig:f2} respectively.

As we can see the the average end-to-end delay slowly increases over time as the
best fit line (linear regression) shows. This is as expected given the fact that
the network is under attack, but the measures taken by the controller seem to
handle this well enough.

The points not shown are not outliers but the times at which a
\texttt{OFPT\_PACKET\_IN} requests are sent to the controller from the switches,
because the flow tables are set to expire after 10 seconds. This points are also
substituted in the linear regression calculation with the average of all the
other data points, since we are interested in the best fit of this other points.

The attacks are performed at time $t_1$ and $t_3$ which is around 11 and 22
seconds after the start of the experiment respectively. Dinamic provisioning
happens at $t_4$ which is at the 33rd seconds on the graph. All of this phases
of the experiment can be clearly seen in figure~\ref{fig:f2}.

The end-to-end delay goes from being mostly below the best fit line, in the
seconds from 0 to 22, to then being above the best fit line and again, at time
33, well below that line.

All the end-to-end delays reported (when the flow tables are installed) are all
sub-millisecond. This is expected given the way in which we sumulated the attack
and given the fact that the entire network is simulated in software.

To conclude our analysis of the results of the experiment, we take a look at the
graph of the utilization of the various links, show in
figure~\ref{fig:utilization}. The graph is fairly busy but if we take as an
example the link between $A$ and $B$ (in the graph reported as \textsf{s2} and
\textsf{s1}) we can see a clear increase in utilization around time 25 as a
consequence of the attack at time $t_3$ (second 22). We can see the same spike
at time 25 in the other 2 graphs.

Again the utilization is always very low. This is because we used
\texttt{ping(1)} to generate traffic in the network which in the worst case has
a 1 Megabit/s bandwidth, which is more than enough to handle that traffic.

\newpage

\begin{figure}
\resizebox{\columnwidth}{!}{\includegraphics{h1}}
\caption{Graph of the end delay over time of $F1$.}
\label{fig:f1}
\end{figure}

\begin{figure}
\resizebox{\columnwidth}{!}{\includegraphics{h2}}
\caption{Graph of the end delay over time of $F2$.}
\label{fig:f2}
\end{figure}

\begin{figure}
\resizebox{\columnwidth}{!}{\rotatebox{0}{\includegraphics{controller}}}
\caption{Graph of the utilization of all links in the network.}
\label{fig:utilization}
\end{figure}

\clearpage

\bibliographystyle{plain}
\bibliography{report}

\end{document}
